\documentclass[a4paper,parskip,headheight=38pt]{scrartcl} % article or scrartcl
\usepackage[utf8]{inputenc}
\usepackage[T1]{fontenc}
\usepackage{amsmath,amssymb,amsfonts}
\usepackage[%
  automark,
  headsepline                %% Separation line below the header
]{scrlayer-scrpage}
\usepackage[english]{babel}
\usepackage{hyphenat}
\usepackage[hidelinks]{hyperref}
\usepackage[top=1.4in, bottom=1.5in, left=1in, right=1in]{geometry}
\usepackage{lastpage}
\usepackage{csquotes}
\usepackage{microtype}
\usepackage{datetime}

\usepackage[normalem]{ulem}
\usepackage{enumerate}
\usepackage{hyperref}

\usepackage{enumitem}

% \usepackage{multicol}
\usepackage{graphicx}
\usepackage{graphics}
% \usepackage{float}
% \usepackage{caption}

\setkomafont{pagehead}{\normalfont\sffamily\footnotesize}
\addtolength{\headheight}{+6pt}
\lohead{Marlene Böhmer, s9meboeh@stud\ldots, 2547718 \\
	Maximilian Köhl, mail@koehlma.de, 2553525 \\
	Maximilian Schwenger, schwenger@stud\ldots, 2542438\\
	Ben Wiederhake, s9bewied@stud\ldots, 2541266}
\rohead{\newline \newline \newline ES16, Set 2, Page {\thepage}/{\pageref*{LastPage}}}

\newtimeformat{mytime}{\twodigit{\THEHOUR}\twodigit{\THEMINUTE}\twodigit{\THESECOND}}
\settimeformat{mytime}
\newdateformat{mydate}{\twodigit{\THEYEAR}\twodigit{\THEMONTH}\twodigit{\THEDAY}}
\cfoot{\tiny\texttt{ID \mydate\today\currenttime}}
\chead{} % Needed because now the \subsections get displayed
\pagestyle{scrheadings}

% \renewcommand{\headrulewidth}{0pt}
% \addtolength{\textheight}{+30mm}
% \addtolength{\textwidth}{+50mm}
% \addtolength{\hoffset}{-7mm}

% \newcommand{\Omicron}{\ensuremath{\mathcal{O}}}
% \newcommand{\omicron}{\ensuremath{o}}
% \newcommand{\set}[1]{\{#1\}}
% \newcommand{\abs}[1]{\lvert #1 \rvert}

\DeclareMathOperator{\sinc}{sinc}

\begin{document}


\setlist[enumerate]{leftmargin=50pt}

\newcommand{\defkind}[2]{
	\newcounter{counter_#2}
	\newcommand{#1}{{\textbf{{#2}\addtocounter{counter_#2}{1}\arabic{counter_#2}}}}
}

\defkind{\musthave}{MR}
\defkind{\nicetohave}{NR}
\defkind{\mustnothave}{N}
\defkind{\plant}{E}
\defkind{\sensors}{S}
\defkind{\actuators}{A}
\defkind{\hardware}{H}
\defkind{\libs}{L}




\section{Overview}
A set of robots that survey a simulated collapsed building (maze), try to find a victim (teddy), and rescue it.

\section{Functional Requirements}
\subsection*{A: Must-Have}
\begin{enumerate}[label=\musthave]
\item while the Tin Bot is operational a green LED shall be on
\item the \enquote{LPS} (local positioning system) needs to supply the E-Pucks with location and orientation data
\item the E-Pucks must share information about the victim
\item chart the inspected environment using the given LPS signal
\item find the victim by…
\begin{enumerate}
\item … exhaustive search, if nothing about the victim is known
\item … attempted shortest path, if the location of the victim is known
\end{enumerate}
\item grab the victim using magnets
\item use proximity sensors / IR receivers to detect whether victim is correctly grabbed
\item bring the victim out
\item while the Tin Bot is escorting the victim, at least the front red LED shall be flashing
\item use the IR-sensors to pick up signals from the victim
\item detect and stay within the borders of the table
\item the surrounding 8 red LEDs shall represent which/whether the IR sensors receive something
\end{enumerate}

\subsection*{B: Nice-to-Have}
\begin{enumerate}[label=\nicetohave]
\item be able to behave reasonably if the LPS is missing
\item export the map
\item be able to behave reasonably if some of the E-Pucks go defunct
\item be able to behave reasonably if the map changes
\item detect if there is no victim
\end{enumerate}

\subsection*{C: Must-not-Have}
\begin{enumerate}[label=\mustnothave]
\item use actual GPS
\item use actual gyroscope
\item do not use camera to identify the victim
\item unsolvable mazes
\item deal with more than one victim
\item do not try to remove obstacles
\item do not try to do detect / react to a malfunctioning / byzantine LPS (although a certain tolerance must be respected of course)
\item detect/react if the victim is lost
\item secure against any kind of attack (physical or non physical)
\end{enumerate}

\section{Non-Functional Requirements}
\subsection*{A: Must-Have}
\begin{enumerate}[label=\musthave]
\item must not be slower than worst-case brute force
\item if there is information do not just run to the position where the information was gathered but instead try to actually localize and find the victim
\item avoid any collisions except with the victim
\item do not make any assumptions about the maze except being solvable
\item the exists of the building are the set of the starting positions of the E-Pucks
\item E-Pucks are equipped with a visual marker on top
\end{enumerate}

\subsection*{B: Nice-to-Have}
\begin{enumerate}[label=\nicetohave]
\item be gentle
\item use the shortest known path out of the building (see starting positions)
\end{enumerate}

\subsection*{C: Must-not-Have}
\begin{enumerate}[label=\mustnothave]
\item deal with uneven floors or multiple floor buildings
\item use actually collapsed buildings or victims
\end{enumerate}


\section{Use-Cases}
\subsection{System Startup}
primary actor: user \\
goal in context: activate the E-Puck(s) \\
precondition: LPS is up and running, E-Puck-batteries are connected and loaded up, E-Puck power switches are in the off position. \\
trigger: power switch of 1 or more E-Pucks is switched on \\
scenario: 
\begin{enumerate}[label={\arabic*.}]
	\item user places E-Pucks on surface 
	\item user switches on the LPS and waits a second 
	\item user switches on the E-Pucks 
\end{enumerate}
exceptions: E-Pucks are too far away; LPS is too far away; E-Pucks cannot properly move on it's own (e.g. placed upside-down) \\

\subsection{System Shutdown}
primary actor: user \\
goal in context: deactivate the simulation / system \\
precondition: At least on E-Puck's power switch is in the on position \\
trigger: power switch of the currently-switched-on E-Pucks are switched off \\
scenario: 
\begin{enumerate}[label={\arabic*.}]
	\item user switches off some of the running E-Pucks 
	\item user switches off the LPS 
	\item user switches off the remaining E-Pucks 
\end{enumerate}

\subsection{Removing an E-Puck}
primary actor: user \\
goal in context: simulate hardware fault in one of the E-Pucks \\
precondition: at least two E-Pucks one the table are switched on and running \\
trigger: power switch of one\footnote{technically: at most all except one} of the switched-on E-Pucks is switched off \\
scenario: 
\begin{enumerate}[label={\arabic*.}]
	\item user selects some E-Pucks to be deactivated (at most all except one) 
	\item user switches off these E-Pucks 
	\item user removes none, some, or all of these E-Pucks (the other switched-off E-Pucks shall be considered \enquote{obstacles}) 
\end{enumerate}
exceptions: only one switched-on E-Puck remains; no E-Pucks are running; switching off and not removing any of the selected E-Pucks would render the maze unsolvable \\

\subsection{Adding/Removing Walls}
primary actor: user \\
goal in context: simulate walls that collapse or break away \\
precondition: N<FIXME> is implemented; E-Pucks are running; if removing a wall, at least one "wall" on the table acting as a wall \\
trigger: user manually removes or adds one or more of the wall elements \\
scenario: 
\begin{enumerate}[label={\arabic*.}]
	\item user removes some wall segment 
	\item user places the removed wall segment somewhere else 
\end{enumerate}
exceptions: adding a wall must not render the maze unsolvable \\

\subsection{Disabling LPS}
primary actor: user \\
goal in context: simulate connectivity problems between LPS and E-Pucks \\
precondition: LPS is up and running \\
trigger: user switches off the LPS (either by software, by obstructing the view of the camera, or by switching off the Raspberry Pi) \\
scenario: 
\begin{enumerate}[label={\arabic*.}]
	\item user literally switches off the Raspberry Pi 
\end{enumerate}

\subsection{Re-enabling LPS}
primary actor: user \\
goal in context: simulate LPS coming back up again (see previous use case) \\
precondition: LPS is not running \\
trigger: user undos the action from the disabling step \\
scenario: 
\begin{enumerate}[label={\arabic*.}]
	\item user switches the Raspberry Pi back on 
\end{enumerate}


\section{Environment properties}

\subsection*{A. Plant}
\begin{enumerate}[label=\plant]
\item victim placed in maze
\item victim sending signals (using IR emitters)
\item victim wears unstylish magnetic belt
\item solvable maze (including table and walls)
\item flat surface
\item initially unblocked view of the camera
\end{enumerate}

\subsection*{B. Sensors}
\begin{enumerate}[label=\sensors]
\item power switches
\item IR receivers
\item proximity sensors
\item Bluetooth receiver
\item camera (LPS)s
\item ground sensors
\end{enumerate}

\subsection*{C. Actuator}
\begin{enumerate}[label=\actuators]
\item blutooth emitter
\item motors (two per E-Puck)
\item IR emitter (victim)
\item LEDs (on LPS and E-Pucks)
\end{enumerate}


\section{Platform}

\subsection*{A. Hardware}
\begin{enumerate}[label=\hardware]
\item $\ge 2$ E-Pucks
\item ATmega (to connect IR sensor with E-Pucks)
\item IR sensors (for E-Pucks)
\item Magnets (for E-Pucks)
\item Raspberry Pi (for LPS)
\item Camera (for LPS)
\item Teddy (a.k.a. victim)
\item Magnet belt (for Teddy)
\item IR emitter (for Teddy)
\item ATtiny (to put unique signal on IR emitter)
\end{enumerate}

\subsection*{B. Libraries}
\begin{enumerate}[label=\libs]
\item E-Puck base libraries
\item Raspberry Pi base system (Linux, \ldots)
\item Raspberry Pi Camera module
\item OpenCV or similar (for image analysis)
\item base libraries for ATmega and ATtiny
\end{enumerate}

\end{document}
