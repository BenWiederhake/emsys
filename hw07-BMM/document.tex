\documentclass[a4paper,parskip,headheight=38pt]{scrartcl} % article or scrartcl
\usepackage[utf8]{inputenc}
\usepackage[T1]{fontenc}
\usepackage{amsmath,amssymb,amsfonts}
\usepackage[%
  automark,
  headsepline                %% Separation line below the header
]{scrlayer-scrpage}
\usepackage[english]{babel}
\usepackage{hyphenat}
\usepackage[hidelinks]{hyperref}
\usepackage[top=1.4in, bottom=1.5in, left=1in, right=1in]{geometry}
\usepackage{lastpage}
\usepackage{csquotes}
\usepackage{microtype}
\usepackage{datetime}

\usepackage[normalem]{ulem}
\usepackage{enumerate}
\usepackage{hyperref}

% \usepackage{multicol}
\usepackage{graphicx}
\usepackage{graphics}
% \usepackage{float}
% \usepackage{caption}

\usepackage{morewrites}
\usepackage[pdf]{graphviz}
\newcommand{\figshrink}{\vspace{-2\baselineskip}}

% \usepackage{marvosym} % \Lightning

\setkomafont{pagehead}{\normalfont\sffamily\footnotesize}
\addtolength{\headheight}{+6pt}
\lohead{Marlene Böhmer, s9meboeh@stud\ldots, 2547718 \\
	Maximilian Köhl, s8makoeh@stud\ldots, 2553525 \\
	Ben Wiederhake, s9bewied@stud\ldots, 2541266}
\rohead{\newline \newline ES16, Set 7, Page {\thepage}/{\pageref*{LastPage}}}

\newtimeformat{mytime}{\twodigit{\THEHOUR}\twodigit{\THEMINUTE}\twodigit{\THESECOND}}
\settimeformat{mytime}
\newdateformat{mydate}{\twodigit{\THEYEAR}\twodigit{\THEMONTH}\twodigit{\THEDAY}}
\cfoot{\tiny\texttt{ID \mydate\today\currenttime}}
\chead{} % Needed because now the \subsections get displayed
\pagestyle{scrheadings}

% \renewcommand{\headrulewidth}{0pt}
% \addtolength{\textheight}{+30mm}
% \addtolength{\textwidth}{+50mm}
% \addtolength{\hoffset}{-7mm}

% \newcommand{\Omicron}{\ensuremath{\mathcal{O}}}
% \newcommand{\omicron}{\ensuremath{o}}
% \newcommand{\set}[1]{\{#1\}}
% \newcommand{\abs}[1]{\lvert #1 \rvert}

% Thanks to https://tex.stackexchange.com/questions/4216/how-to-typeset-correctly
\newcommand{\defeq}{\mathrel{\vcenter{\baselineskip0.5ex \lineskiplimit0pt
                    \hbox{\scriptsize.}\hbox{\scriptsize.}}}%
                    =}

\newcommand{\ltlbinop}[1]{\mathbin{\ensuremath{\mathbf{{#1}}}}}
\newcommand{\ltlunop}[1]{\mathopen{\ensuremath{\mathbf{{#1}}}}\,}
    % \ltlunop:
    % Although the appropriate class is actually \mathop instead of \mathopen,
    % \mathop would change the vertical alignment.  Sigh.
    % Source: http://tex.stackexchange.com/a/38984
\newcommand{\ltlG}{\ltlunop{G}}
\newcommand{\ltlF}{\ltlunop{F}}
\newcommand{\ltlX}{\ltlunop{X}}
\newcommand{\ltlU}{\ltlbinop{U}}
\newcommand{\ltlUm}{\ltlbinop{Um}}

%\DeclareMathOperator{\sinc}{sinc}

\begin{document}

\section*{Problem 1: Reactive Systems \sout{Group}}

\subsection*{Part (a)}

As usual, we define the operator \enquote{$\implies$} the obvious way.
% Namely: $(a \implies b) \iff \neg(a \land \neg b) \lor State(Cthulu) = \mathtt{CTHULU\_STATE\_ATTACKING}$

Furthermore, $b$ shall be the condition \enquote{the user is pressing a
button}, and $h$ shall mean \enquote{the horn is ringing}.

\[ \ltlG \big(  (b \implies \ltlF h) \land \neg (h \land \ltlX h) \big) \]

\subsection*{Part (b)}

\digraph[width=\textwidth]{p1b}{
    init [style=invis,label="",shape=none];
    nothing [label="nothing", style=bold];
    not [label="not ringing yet"];
    ring [label="ringing",style=bold];
    error [label="error"];
    init -> nothing;
    nothing -> nothing [label="!b && !h"];
    nothing -> not [label="b && !h"];
    nothing -> ring [label="h"];
    not -> ring [label="h"];
    not -> not [label="!h"];
    ring -> not [label="b && !h"];
    ring -> nothing [label="!b && !h"];
    ring -> error [label="h"];
    error -> error [label="true"];
}

FIXME: Is this the kind of monitor they were asking for?

\subsection*{Part (c)}

FIXME

\subsection*{Part (d)}

FIXME


\section*{Problem 2: Timed Games}

FIXME


\end{document}
