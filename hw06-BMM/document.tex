\documentclass[a4paper,parskip,headheight=38pt]{scrartcl} % article or scrartcl
\usepackage[utf8]{inputenc}
\usepackage[T1]{fontenc}
\usepackage{amsmath,amssymb,amsfonts}
\usepackage[%
  automark,
  headsepline                %% Separation line below the header
]{scrlayer-scrpage}
\usepackage[english]{babel}
\usepackage{hyphenat}
\usepackage[hidelinks]{hyperref}
\usepackage[top=1.4in, bottom=1.5in, left=1in, right=1in]{geometry}
\usepackage{lastpage}
\usepackage{csquotes}
\usepackage{microtype}
\usepackage{datetime}

\usepackage[normalem]{ulem}
\usepackage{enumerate}
\usepackage{hyperref}

% \usepackage{multicol}
\usepackage{graphicx}
\usepackage{graphics}
% \usepackage{float}
% \usepackage{caption}

\usepackage{marvosym} % \Lightning

\setkomafont{pagehead}{\normalfont\sffamily\footnotesize}
\addtolength{\headheight}{+6pt}
\lohead{Marlene Böhmer, s9meboeh@stud\ldots, 2547718 \\
	Maximilian Köhl, s8makoeh@stud\ldots, 2553525 \\
	Ben Wiederhake, s9bewied@stud\ldots, 2541266}
\rohead{\newline \newline ES16, Set 6, Page {\thepage}/{\pageref*{LastPage}}}

\newtimeformat{mytime}{\twodigit{\THEHOUR}\twodigit{\THEMINUTE}\twodigit{\THESECOND}}
\settimeformat{mytime}
\newdateformat{mydate}{\twodigit{\THEYEAR}\twodigit{\THEMONTH}\twodigit{\THEDAY}}
\cfoot{\tiny\texttt{ID \mydate\today\currenttime}}
\chead{} % Needed because now the \subsections get displayed
\pagestyle{scrheadings}

% \renewcommand{\headrulewidth}{0pt}
% \addtolength{\textheight}{+30mm}
% \addtolength{\textwidth}{+50mm}
% \addtolength{\hoffset}{-7mm}

% \newcommand{\Omicron}{\ensuremath{\mathcal{O}}}
% \newcommand{\omicron}{\ensuremath{o}}
% \newcommand{\set}[1]{\{#1\}}
% \newcommand{\abs}[1]{\lvert #1 \rvert}

% Thanks to https://tex.stackexchange.com/questions/4216/how-to-typeset-correctly
\newcommand{\defeq}{\mathrel{\vcenter{\baselineskip0.5ex \lineskiplimit0pt
                    \hbox{\scriptsize.}\hbox{\scriptsize.}}}%
                    =}

%\DeclareMathOperator{\sinc}{sinc}

\begin{document}

\section*{Problem 1: Testing}

\subsection*{Part (a)}

\begin{description}
    \item[Statement, $t_0$:] Not satisfied, 3 is missing.
    \item[Statement, $t_1$:] Not satisfied, 5 is missing.
    \item[Statement, both:] Satisfied.
    \item[Decision, $t_0$:] Not satisfied, decision $2 \rightarrow 3$ is missing.
    \item[Decision, $t_1$:] Not satisfied, decision $4 \rightarrow 5$ is missing.
    \item[Decision, both:] Satisfied.
\end{description}

\subsection*{Part (b)}

Assuming that 1 and 7 are supposed to be the initial and final nodes, respectively:

\[ t_2 = [1, 2, 3, 2, 4, 5, 6, 1, 7] \]

Any test path necessarily covers both decisions made in 1 and both
decisions made in 2.  Thus, any further test path is \emph{basically}
the same as $t_2$, but not necessarily identical.  For example, this is
\enquote{another one}:

\[ t_3 = [1, 2, 3, 2, 4, 5, 6, 1, 2, 4, 5, 6, 1, 7] \]

\subsection*{Part (c)}

FIXME


\section*{Problem 2: Modeling with Timed Automata}

FIXME


\section*{Problem 3: Region Automaton}

FIXME


\section*{Problem 4: LTL}

FIXME


\end{document}
