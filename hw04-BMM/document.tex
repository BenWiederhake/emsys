\documentclass[a4paper,parskip,headheight=38pt]{scrartcl} % article or scrartcl
\usepackage[utf8]{inputenc}
\usepackage[T1]{fontenc}
\usepackage{amsmath,amssymb,amsfonts}
\usepackage[%
  automark,
  headsepline                %% Separation line below the header
]{scrlayer-scrpage}
\usepackage[english]{babel}
\usepackage{hyphenat}
\usepackage[hidelinks]{hyperref}
\usepackage[top=1.4in, bottom=1.5in, left=1in, right=1in]{geometry}
\usepackage{lastpage}
\usepackage{csquotes}
\usepackage{microtype}
\usepackage{datetime}

\usepackage[normalem]{ulem}
\usepackage{enumerate}
\usepackage{hyperref}

% \usepackage{multicol}
\usepackage{graphicx}
\usepackage{graphics}
% \usepackage{float}
% \usepackage{caption}

\setkomafont{pagehead}{\normalfont\sffamily\footnotesize}
\addtolength{\headheight}{+6pt}
\lohead{Marlene Böhmer, s9meboeh@stud\ldots, 2547718 \\
	Maximilian Köhl, s8makoeh@stud\ldots, 2553525 \\
	Ben Wiederhake, s9bewied@stud\ldots, 2541266}
\rohead{\newline \newline ES16, Set 4, Page {\thepage}/{\pageref*{LastPage}}}

\newtimeformat{mytime}{\twodigit{\THEHOUR}\twodigit{\THEMINUTE}\twodigit{\THESECOND}}
\settimeformat{mytime}
\newdateformat{mydate}{\twodigit{\THEYEAR}\twodigit{\THEMONTH}\twodigit{\THEDAY}}
\cfoot{\tiny\texttt{ID \mydate\today\currenttime}}
\chead{} % Needed because now the \subsections get displayed
\pagestyle{scrheadings}

% \renewcommand{\headrulewidth}{0pt}
% \addtolength{\textheight}{+30mm}
% \addtolength{\textwidth}{+50mm}
% \addtolength{\hoffset}{-7mm}

% \newcommand{\Omicron}{\ensuremath{\mathcal{O}}}
% \newcommand{\omicron}{\ensuremath{o}}
% \newcommand{\set}[1]{\{#1\}}
% \newcommand{\abs}[1]{\lvert #1 \rvert}

% Thanks to https://tex.stackexchange.com/questions/4216/how-to-typeset-correctly
\newcommand{\defeq}{\mathrel{\vcenter{\baselineskip0.5ex \lineskiplimit0pt
                    \hbox{\scriptsize.}\hbox{\scriptsize.}}}%
                    =}

\DeclareMathOperator{\sinc}{sinc}

\begin{document}

\section*{Problem 1: Hardware Redundancy}
\begin{enumerate}
\item The basic idea of static hardware redundancy is to have multiple copies of the same component in place that all do the same calculation in parallel. Which value is the \enquote{right} one is decided by majority vote and independent of the actual computation.

The basic idea of dynamic hardware redundancy is to have multiple copies of the same component in place, they may or may not do the same calculation in parallel. Whether a value, computed by the default component is \enquote{right} is decided by some other component which switches to one of the other standby components when it detects a fault.

\item Static: Random faults if the reliability of the components is high enough. Dynamic: Only faults that are detectable by the system (under the assumption that at least one of the standby units gives correct results).

\item Static: The reliability and amount of the components, and the reliability of the voter.

Dynamic: The reliability of the components, and the ability (set of faults that can be detected) and reliability fault detector.

\item To apply static hardware redundancy one does not need a sophisticated fault detection algorithm/component. Therefore they are easier to implement. The reliability of a voter is in general higher than the reliability of a fault detector. However, the cost might be higher, due to the sheer amount of the main component.

With dynamic hardware redundancy you may have cold spares which means they do not wear out as fast as the default component. With static hardware redundancy all components are exposed to the same stress level leading to similar wear-out rates.
\end{enumerate}


\section*{Problem 2: Static Redundancy}
\begin{enumerate}
\item \begin{align*}
R_{TMR}(t) = 1 - \overbrace{\left( \underbrace{3 \cdot (1 - R(t))^2 \cdot R(t)}_{\text{(a)}} + \underbrace{(1 - R(t))^2}_{\text{(b)}} \right)}^{\text{failure probability}}
\end{align*}
The system fails if (a) two components fail and one remains functional. There are three combinations that result in such a scenario. Or (b) all three components fail.
\begin{align*}
R_{TMR}(t) &= 1 - (3R(t) - 6R(t)^2 + 3R(t)^3 + 1 - 3R(t) + 3R(t)^2 - R(t)^3) \\
&= 3 R^2(t) - 2 R^3(t)
\end{align*}
\item
No it is not always higher. Proof by counterexample for $R(t) = \frac{1}{2}$:
\begin{align*}
& 3\left(\frac{1}{2}\right)^2 - 2\left(\frac{1}{2}\right)^3 \\
=& \frac{3}{4} - \frac{2}{8} \\
=& \frac{1}{2} \\
\not >& R(t)
\end{align*}
For all $t$ with $R(t) \leq \frac{1}{2}$ the TMR arrangement is not justified.
\item With 2. we get:
\begin{align*}
\frac{1}{2} &\geq e^{-\lambda t} \\
\ln \frac{1}{2} &\geq -\lambda t \\
t &\geq \frac{\ln 2}{\lambda}
\end{align*}
\end{enumerate}


\section*{Problem 3: Reliability Analysis}

\subsection*{Part 1}

Components 5, 6, and 7 can be treated as a single component \enquote{8} with a reliability of:
 %
\[R(8) = 1 - (1 - R(5))(1 - R(6))(1 - R(7)) = 1 - (1-0.7)^3 = 0.973\]

Components 2 and 8 can be treated as a single component \enquote{9} with a reliability of
 %
\[R(9) = R(2) \cdot R(8) = 0.9 \cdot 0.973 = 0.8757\]

Components 9 and 3 can be treated as a single component \enquote{10} with a reliability of
 %
\[R(10) = 1 - (1 - R(9))(1 - R(3)) = 1 - 0.0294 = 0.9751\]

Components 1 and 10 can be treated as a single component \enquote{11} with a reliability of
 %
\[R(11) = 0.95 \cdot 0.9751 = 0.9263\]

Components 11 and 4 can be treated as a single component \enquote{S} with a reliability of
 %
\[R(S) = 1 - 0.0737 \cdot 0.15 = 0.989\]

Component S represents the system as a whole, therefore the overall reliability is exactly $0.989$.

\subsection*{Part 2}

By working from right to left:
 %
\[\{  \{5,6,7,3,4\}, \{2,3,4\}, \{1,4\}  \}\]

\subsection*{Part 3}

As a shorthand-notation, let $NR_{\alpha}(t) \defeq \prod_{i \in \alpha} (1-R_{i}(t))$

\begin{align*}
    R(t)
    &> 1-\left( NR_{5,6,7,3,4}(t) + NR_{2,3,4}(t) + NR_{1,4}(t) \right) \\
    &= 1-\left( 0.00081 + 0.003 + 0.0075 \right) \\
    &= 1 - 0.0113 \\
    &= 0.9887
\end{align*}

In comparison, it's clear that this is accurate, but not always tight.  Most
importantly, we don't see any justification to use an exponential-time
approximation when there's a linear-time exact algorithm.


\section*{Problem 4: Communication}
The static segment implements time-triggered behavior and by that guarantees a specific latency for time critical events. The dynamic segment improves the overall bandwidth by not wasting time slots compared to the static segment. They are combined in order to enjoy the benefits of both approaches.

Jitter means a variation in latency. Glitches are bit-flips, which means that some bits have the wrong value. If there are bit-flips and the system detects and corrects them, this might result in a variation of latency.

\end{document}
