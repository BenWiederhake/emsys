\documentclass[a4paper,parskip,headheight=38pt]{scrartcl} % article or scrartcl
\usepackage[utf8]{inputenc}
\usepackage[T1]{fontenc}
\usepackage{amsmath,amssymb,amsfonts}
\usepackage[%
  automark,
  headsepline                %% Separation line below the header
]{scrlayer-scrpage}
\usepackage[english]{babel}
\usepackage{hyphenat}
\usepackage[hidelinks]{hyperref}
\usepackage[top=1.4in, bottom=1.5in, left=1in, right=1in]{geometry}
\usepackage{lastpage}
\usepackage{csquotes}
\usepackage{microtype}
\usepackage{datetime}

\usepackage[normalem]{ulem}
\usepackage{enumerate}
\usepackage{hyperref}

% \usepackage{multicol}
\usepackage{graphicx}
\usepackage{graphics}
% \usepackage{float}
% \usepackage{caption}

\setkomafont{pagehead}{\normalfont\sffamily\footnotesize}
\addtolength{\headheight}{+6pt}
\lohead{Marlene Böhmer, s9meboeh@stud\ldots, 2547718 \\
	Maximilian Köhl, s8makoeh@stud\ldots, 2553525 \\
	Ben Wiederhake, s9bewied@stud\ldots, 2541266}
\rohead{\newline \newline ES16, Set 4, Page {\thepage}/{\pageref*{LastPage}}}

\newtimeformat{mytime}{\twodigit{\THEHOUR}\twodigit{\THEMINUTE}\twodigit{\THESECOND}}
\settimeformat{mytime}
\newdateformat{mydate}{\twodigit{\THEYEAR}\twodigit{\THEMONTH}\twodigit{\THEDAY}}
\cfoot{\tiny\texttt{ID \mydate\today\currenttime}}
\chead{} % Needed because now the \subsections get displayed
\pagestyle{scrheadings}

% \renewcommand{\headrulewidth}{0pt}
% \addtolength{\textheight}{+30mm}
% \addtolength{\textwidth}{+50mm}
% \addtolength{\hoffset}{-7mm}

% \newcommand{\Omicron}{\ensuremath{\mathcal{O}}}
% \newcommand{\omicron}{\ensuremath{o}}
% \newcommand{\set}[1]{\{#1\}}
% \newcommand{\abs}[1]{\lvert #1 \rvert}

% Thanks to https://tex.stackexchange.com/questions/4216/how-to-typeset-correctly
\newcommand{\defeq}{\mathrel{\vcenter{\baselineskip0.5ex \lineskiplimit0pt
                    \hbox{\scriptsize.}\hbox{\scriptsize.}}}%
                    =}

\DeclareMathOperator{\sinc}{sinc}

\begin{document}

\section*{Problem 1: Hardware Redundancy}

FIXME


\section*{Problem 2: Static Redundancy}

FIXME


\section*{Problem 3: Reliability Analysis}

\subsection*{Part 1}

Components 5, 6, and 7 can be treated as a single component \enquote{8} with a reliability of:
 %
\[R(8) = 1 - (1 - R(5))(1 - R(6))(1 - R(7)) = 1 - (1-0.7)^3 = 0.973\]

Components 2 and 8 can be treated as a single component \enquote{9} with a reliability of
 %
\[R(9) = R(2) \cdot R(8) = 0.9 \cdot 0.973 = 0.8757\]

Components 9 and 3 can be treated as a single component \enquote{10} with a reliability of
 %
\[R(10) = 1 - (1 - R(9))(1 - R(3)) = 1 - 0.0294 = 0.9751\]

Components 1 and 10 can be treated as a single component \enquote{11} with a reliability of
 %
\[R(11) = 0.95 \cdot 0.9751 = 0.9263\]

Components 11 and 4 can be treated as a single component \enquote{S} with a reliability of
 %
\[R(S) = 1 - 0.0737 \cdot 0.15 = 0.989\]

Component S represents the system as a whole, therefore the overall reliability is exactly $0.989$.

\subsection*{Part 2}

By working from right to left:
 %
\[\{  \{5,6,7,3,4\}, \{2,3,4\}, \{1,4\}  \}\]

\subsection*{Part 3}

As a shorthand-notation, let $NR_{\alpha}(t) \defeq \prod_{i \in \alpha} (1-R_{i}(t))$

\begin{align*}
    R(t)
    &> 1-\left( NR_{5,6,7,3,4}(t) + NR_{2,3,4}(t) + NR_{1,4}(t) \right) \\
    &= 1-\left( 0.00081 + 0.003 + 0.0075 \right) \\
    &= 1 - 0.0113 \\
    &= 0.9887
\end{align*}

In comparison, it's clear that this is accurate, but not always tight.  Most
importantly, we don't see any justification to use an exponential-time
approximation when there's a linear-time exact algorithm.


\section*{Problem 4: Communication}

FIXME


\end{document}
